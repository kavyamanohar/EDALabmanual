\chapter [Solving Equations]{Solving Equations}


\section*{AIM}
\begin{itemize}
\item
To solve a linear equation in two variables accepting coefficients from the user.
\item 
To solve quadratic equation accespting coefficients from the user.

\end{itemize}

\section*{THEORY}
\paragraph{}

Linear equations can be solved using $linsolve()$ function.

\paragraph{}

\section*{PROCEDURE}

\paragraph{}
\begin{enumerate}
\item
Start Scilab on PC and Scilab console window opens. Create a new blank SciNote.
\item
The code for the required program is typed and saved as Scilab SCE file with an extension .sci
\item
The program is run using the “execute”.
\item
On executing the program in listing \ref{linEqn}, the following prompt appears on the Scilab console.

\begin{lstlisting}[numbers=none]
Enter the coefficients of first equation(Enter a1.x+b1.y=c1 as [a1 b1 -c1]:
\end{lstlisting}

For the equation $3x+2y=1$, input [3 2 -1] as shown below and press  `Enter'.
\begin{lstlisting}[numbers=none]
Enter the coefficients of first equation(Enter a1.x+b1.y=c1 as [a1 b1 -c1]:[3 2 -1] 
\end{lstlisting}
Enter the values of coeffiecients of second equation when prompted as
\begin{lstlisting}[numbers=none]
Enter the coefficients of second equation(Enter a2.x+b2.y=c2 as [a2 b2 -c2]:
\end{lstlisting}
For the equation $2x+4y=5$, input [2 4 -5] as shown below and press  `Enter'.
\begin{lstlisting}[numbers=none]
Enter the coefficients of second equation(Enter a2.x+b2.y=c2 as [a2 b2 -c2]:[2 4 -5]
\end{lstlisting}
\item
THe following prompt appears on the console on executing code in listing \ref{quadEqn}
\begin{lstlisting}[numbers=none]
Solving Quadratic Equations   
 
 ----------------------------   
 
     
Enter the coefficients(Eg: To solve c+(b*x)+a(x^2)=0, enter [c b a]):
\end{lstlisting}

For the equation $3+5x+6x^2=0$, input [3 5 6] and press `Enter'.
\item

The results and the errors in the program are displayed in the console window.
\end{enumerate}

\section*{SCILAB CODE}
\subsection*{Solve Linear equations in two variables}


\lstinputlisting[caption={Code to solve linear equation in two variables},label={linEqn}]{./scilabCode/LinearEquation.sci}

\subsection{Solve quadratic equation}

\lstinputlisting[caption={Code to solve quadratic equation},label={quadEqn}]{./scilabCode/quadEquation.sci}

\section*{RESULT}

For the program in listing \ref{linEqn} and the inputs as defined in procedure 4, following will be the final result.
\begin{lstlisting}[numbers=none]
The solutions are x=-0.750 and y=1.625
\end{lstlisting}



For the program in listing \ref{quadEqn} and the inputs as defined in procedure 4, following will be the final result.
\begin{lstlisting}[numbers=none]
1st root of the equation is =   
 
  - 0.4166667 + 0.5713046i  
 
 2nd root of the equation is   
 
  - 0.4166667 - 0.5713046i  
 
 The roots found using built in roots() function:   
 
  - 0.4166667 + 0.5713046i  
  - 0.4166667 - 0.5713046i
\end{lstlisting}


