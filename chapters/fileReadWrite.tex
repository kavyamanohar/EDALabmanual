\chapter [Reading and writing a text file]{Reading and writing a text file}


\section*{Problem Definition}

The requirement is to read a set of floating point numbers from a formatted text file sort them in ascending order and write the sorted numbers into a new text file.

\section*{Background details}
\paragraph{}


It often requires to read data from an existing file, process the data and write the result into a new file. There are various inbuilt functions to carryout file operations.

\paragraph{}

\section*{Let us experiment}

\paragraph{}
\begin{enumerate}
\item
Start Scilab on PC and Scilab console window opens. Create a new blank SciNote.
\item
The code for the required program is typed and saved as Scilab SCE file with an extension .sci
\item
The program is run using the “execute”. Make sure the formattedd text file `text.txt' resides in the same directory as that of the program.
\item
On executing the program in listing \ref{fileReadWrite}, the results are displayed in the console window.The result is also written to a text file `newtext.txt'.
\end{enumerate}

\section*{SCILAB CODE}
\subsection*{Solve Linear equations in two variables}


\lstinputlisting[caption={Code to do read and write file operations },label={fileReadWrite}]{./scilabCode/fileReadWrite.sci}

\lstinputlisting[caption={text.txt(Unsorted File)}]{./scilabCode/text.txt}


\section*{RESULT}

For the program in listing \ref{fileReadWrite} following will be the final result.

\begin{lstlisting}[numbers=none]
 Scilab can read selected file
The floating point numbers from a formated file
 
    33.220001  
    462.       
    27.110001  
    383.22     
Values in assending order copyed in another text file with name newtext in same directory
 
    27.110001  
    33.220001  
    383.22     
    462.       
 \end{lstlisting}
 
 \lstinputlisting[caption={newtext.txt(Sorted File)}]{./scilabCode/text.txt}
